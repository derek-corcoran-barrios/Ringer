\documentclass[]{elsarticle} %review=doublespace preprint=single 5p=2 column
%%% Begin My package additions %%%%%%%%%%%%%%%%%%%
\usepackage[hyphens]{url}
\usepackage{lineno} % add
\providecommand{\tightlist}{%
  \setlength{\itemsep}{0pt}\setlength{\parskip}{0pt}}

\bibliographystyle{elsarticle-harv}
\biboptions{sort&compress} % For natbib
\usepackage{graphicx}
\usepackage{booktabs} % book-quality tables
%% Redefines the elsarticle footer
%\makeatletter
%\def\ps@pprintTitle{%
% \let\@oddhead\@empty
% \let\@evenhead\@empty
% \def\@oddfoot{\it \hfill\today}%
% \let\@evenfoot\@oddfoot}
%\makeatother

% A modified page layout
\textwidth 6.75in
\oddsidemargin -0.15in
\evensidemargin -0.15in
\textheight 9in
\topmargin -0.5in
%%%%%%%%%%%%%%%% end my additions to header

\usepackage[T1]{fontenc}
\usepackage{lmodern}
\usepackage{amssymb,amsmath}
\usepackage{ifxetex,ifluatex}
\usepackage{fixltx2e} % provides \textsubscript
% use upquote if available, for straight quotes in verbatim environments
\IfFileExists{upquote.sty}{\usepackage{upquote}}{}
\ifnum 0\ifxetex 1\fi\ifluatex 1\fi=0 % if pdftex
  \usepackage[utf8]{inputenc}
\else % if luatex or xelatex
  \usepackage{fontspec}
  \ifxetex
    \usepackage{xltxtra,xunicode}
  \fi
  \defaultfontfeatures{Mapping=tex-text,Scale=MatchLowercase}
  \newcommand{\euro}{€}
\fi
% use microtype if available
\IfFileExists{microtype.sty}{\usepackage{microtype}}{}
\ifxetex
  \usepackage[setpagesize=false, % page size defined by xetex
              unicode=false, % unicode breaks when used with xetex
              xetex]{hyperref}
\else
  \usepackage[unicode=true]{hyperref}
\fi
\hypersetup{breaklinks=true,
            bookmarks=true,
            pdfauthor={},
            pdftitle={Have coaches changed how they select which players to give more minutes to?},
            colorlinks=true,
            urlcolor=blue,
            linkcolor=magenta,
            pdfborder={0 0 0}}
\urlstyle{same}  % don't use monospace font for urls
\setlength{\parindent}{0pt}
\setlength{\parskip}{6pt plus 2pt minus 1pt}
\setlength{\emergencystretch}{3em}  % prevent overfull lines
\setcounter{secnumdepth}{0}
% Pandoc toggle for numbering sections (defaults to be off)
\setcounter{secnumdepth}{0}
% Pandoc header


\usepackage[nomarkers]{endfloat}

\begin{document}
\begin{frontmatter}

  \title{Have coaches changed how they select which players to give more minutes
to?}
    \author[Pontificia Universidad Catolica de Chile]{Derek Corcoran\corref{c1}}
   \ead{derek.corcoran.barrios@gmail.com} 
   \cortext[c1]{Corresponding Author}
    \author[The University of Mississippi]{Nicholas M. Watanabe}
   \ead{nmwatana@olemiss.edu} 
  
      \address[Some Institute of Technology]{Department, Street, City, State, Zip}
    \address[Another University]{Department, Street, City, State, Zip}
  
  \begin{abstract}
  Since the NBA adopted the three point line in .
  
  It consists of two paragraphs.
  \end{abstract}
  
 \end{frontmatter}

\emph{Text based on elsarticle sample manuscript, see
\url{http://www.elsevier.com/author-schemas/latex-instructions\#elsarticle}}

\section{The Elsevier article class}\label{the-elsevier-article-class}

\paragraph{Installation}\label{installation}

If the document class \emph{elsarticle} is not available on your
computer, you can download and install the system package
\emph{texlive-publishers} (Linux) or install the LaTeX~package
\emph{elsarticle} using the package manager of your TeX~installation,
which is typically TeX~Live or MikTeX.

\paragraph{Usage}\label{usage}

Once the package is properly installed, you can use the document class
\emph{elsarticle} to create a manuscript. Please make sure that your
manuscript follows the guidelines in the Guide for Authors of the
relevant journal. It is not necessary to typeset your manuscript in
exactly the same way as an article, unless you are submitting to a
camera-ready copy (CRC) journal.

\subsection{Methods}\label{methods}

All variables were scaled and centered (Bro and Smilde 2003) using the
caret package (Kuhn and Johnson 2013). We performed all of the analyses
using R statistical Software (R Core Team 2016). We fitted the occupancy
models were using the unmarked package (Fiske and Chandler 2011) and we
ranked the models based on Akaike's Information Criteria for small
sample sizes (AICc) using the MuMin Package (Bartoń 2016). First, we
selected the best detection probability model for each species by
fitting all possible first order models while holding occupancy
constant. Next, the detection portion of the model was fixed while the
occupancy component of the model was selected following similar steps.
Since both burn intensity measurements were highly correlated
(\textgreater{}= 0.98 Pearson correlation coefficient), we did not allow
those variables to coexist in the tested models. We expected non-linear
relationships between fire intensity and occupancy, and thus we included
both the linear and quadratic parameter of the burn intensity variables.
We didn't use model averaging since even though collinear variables were
prohibited to coexist in the same model, these might coexist in the
average model (Cade 2015). We projected the resulting occupancy model to
the entire National Forest for each species included in this study and
the best model for each species was assessed for goodness of fit using a
Pearson Chi square statistic (MacKenzie and Bailey 2004). For each bat
species, occupancy inside and outside of the fire areas was compared by
with a t-test between the fitted values for the sampled points.

\paragraph{Functionality}\label{functionality}

The Elsevier article class is based on the standard article class and
supports almost all of the functionality of that class. In addition, it
features commands and options to format the

\begin{itemize}
\item
  document style
\item
  baselineskip
\item
  front matter
\item
  keywords and MSC codes
\item
  theorems, definitions and proofs
\item
  lables of enumerations
\item
  citation style and labeling.
\end{itemize}

\section{Front matter}\label{front-matter}

The author names and affiliations could be formatted in two ways:

\begin{enumerate}
\def\labelenumi{(\arabic{enumi})}
\item
  Group the authors per affiliation.
\item
  Use footnotes to indicate the affiliations.
\end{enumerate}

See the front matter of this document for examples. You are recommended
to conform your choice to the journal you are submitting to.

\section{Bibliography styles}\label{bibliography-styles}

There are various bibliography styles available. You can select the
style of your choice in the preamble of this document. These styles are
Elsevier styles based on standard styles like Harvard and Vancouver.
Please use BibTeX~to generate your bibliography and include DOIs
whenever available.

Here are two sample references: Barto{ń} (2013; Cade 2015).

\section*{References}\label{references}
\addcontentsline{toc}{section}{References}

Barto{ń}, Kamil. 2013. ``MuMIn: Multi-Model Inference. R Package Version
1.9. 13.'' \emph{The Comprehensive R Archive Network (CRAN), Vienna,
Austria}.

Bro, Rasmus, and Age K Smilde. 2003. ``Centering and Scaling in
Component Analysis.'' \emph{Journal of Chemometrics} 17 (1). Wiley
Online Library: 16--33.

Cade, Brian S. 2015. ``Model Averaging and Muddled Multimodel
Inferences.'' \emph{Ecology} 96 (9). Wiley Online Library: 2370--82.

Kuhn, Max, and Kjell Johnson. 2013. \emph{Applied Predictive Modeling}.
Vol. 26. Springer.

\end{document}


